\section{Introduction}
We shall start with a background of the classical computations and then develop quantum methods for the same.
\subsection{Background}
Blockchain is the method used at present for secure peer to peer transactions through decentralized currencies\cite{btc}. It comprises peers broadcasting transactions on a decentralized ledger, secured by a digital signature. The signature can be generated uniquely by using a private key held by the node generating the signature only. The other peers can verify the authenticity of the signature using a widely available public key. Since copies of ledgers are possessed by all the peers on the network, it is imperative to develop a form of authenticity check on the ledgers. This is doing by using a special string called the nonce, which is in turn dependent on the contents of the ledger, to generate a hash satisfying some conditions. Mathematics show that brute force search is the only method for generating such a nonce. The probability of success is of the order of $\bfrac{1}{2^N}$, where $N$ is the number of bits used for hashing. For a standard 256-bit hash, the probability of finding a special nonce is too low for all practical purposes. Thus, "guessing" a nonce is nearly next to impossible at an individual level. To increase the security even further, blocks of transactions with a valid nonce are linked together in the form of a chain to prevent any fraudulent manipulation of previous transactions. This is called a block chain.
Quantum computing is a form of computing that uses the quantum mechanical states of particles, called qubits, to store information, instead of storing them on classical switches. Two special properties of qubits are that they can undergo superposition and entanglement. These properties allow special quantum algorithms to be developed, which speed an otherwise slow classical process. Quantum computers are speculated to posses serious threats to the classical computing algorithms\cite{enc_br}, including those used in the security of bitcoins. Quantum algorithms like Grover's algorithm and a more generalized version of it can search for the nonce much faster than a classical computer.
\subsection{Objectives And Applications}
The aim of the project is to find loopholes in the present system of cryptocurrency\cite{btc} when subjected to quantum computation and to find methods to improve the security. The problem at hand involves understanding the current methods of encryption and hashing, following by designing a quantum algorithm to break the same. At the last leg of the project, we shall solve these problems by using quantum algorithms for hashing.

Our work may be used to develop and enhance secure quantum algorithms for crypto-currencies. The algorithms presented in our work can also be used to design a new crypto-currency system on a real quantum computer back-end.
\subsection{System Requirements}
The code for our work uses Qiskit on a Jupyter Notebook on Python. The following are the minimum system requirements to run our code.\\
Processors: Intel Core™ i3 processor or higher\\
Disk space: 1 GB or higher\\
Operating systems: Windows 7 or later, macOS 10.12.6 or later, and Ubuntu 16.04 or later\\
Python: 3.6 or higher\\
Numpy: 1.20.0 or higher\\
Jupyter Lab/Notebook: 6.3.0 or higher\\
%The code is available \href{https://github.com/Ankan-Mukherjee/NIUS.git}{here}.